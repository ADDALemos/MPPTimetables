%%%%%%%%%%%%%%%%%%%%%%% file template.tex %%%%%%%%%%%%%%%%%%%%%%%%%
%
% This is a general template file for the LaTeX package SVJour3
% for Springer journals.          Springer Heidelberg 2010/09/16
%
% Copy it to a new file with a new name and use it as the basis
% for your article. Delete % signs as needed.
%
% This template includes a few options for different layouts and
% content for various journals. Please consult a previous issue of
% your journal as needed.
%
%%%%%%%%%%%%%%%%%%%%%%%%%%%%%%%%%%%%%%%%%%%%%%%%%%%%%%%%%%%%%%%%%%%
\RequirePackage{fix-cm}
%
\documentclass{svjour3}                     % onecolumn (standard format)
%\documentclass[smallcondensed]{svjour3}     % onecolumn (ditto)
%\documentclass[smallextended]{svjour3}       % onecolumn (second format)
%\documentclass[twocolumn]{svjour3}          % twocolumn
%
\smartqed  % flush right qed marks, e.g. at end of proof
%
\usepackage{graphicx}
\usepackage[acronym]{glossaries}

%
% \usepackage{mathptmx}      % use Times fonts if available on your TeX system
%
% insert here the call for the packages your document requires
%\usepackage{latexsym}
% etc.
%
% please place your own definitions here and don't use \def but
% \newcommand{}{}
%
% Insert the name of "your journal" with
% \journalname{myjournal}
%
\begin{document}

\title{ITC-2019: An Integer Linear Programming based approach to solve University Timetabling problems \thanks{This work was supported by Universidade de Lisboa, Instituto Superior T\'ecnico and Departamento de Engenharia Inform\'atica (DEI) and by national funds through Funda\c{c}\~ao para a Ci\^encia e a Tecnologia (FCT) with reference SFRH/BSAB/143643/2019 (sabbatical grant) and UID/CEC/50021/2019 (INESC-ID multi-annual funding).} }
%\subtitle{Do you have a subtitle?\\ If so, write it here}

\titlerunning{ITC-2019: An ILP based approach to solve University Timetabling problems}        % if too long for running head

\author{Alexandre Lemos \and
Pedro T. Monteiro \and  In\^es Lynce}
%

% First names are abbreviated in the running head.
% If there are more than two authors, 'et al.' is used.
%
\institute{A. Lemos \and P.T. Monteiro \and I. Lynce  \at Instituto Superior T\'ecnico, Universidade de Lisboa\\ INESC-ID,
Rua Alves Redol, 9, 1000-029 Lisboa, Portugal\\
\email{\{alexandre.lemos,pedro.tiago.monteiro,ines.lynce\}\\@tecnico.ulisboa.pt}
}

\date{Received: date / Accepted: date}
% The correct dates will be entered by the editor


\maketitle

\begin{abstract}
This paper describes an algorithm that uses an ILP based encoding to solve course timetabling and student sectitiong problems as specified for the International Timetabling Competition 2019. We developed an hybridize algorithm that combines local search and an Integer programming solver.
\keywords{ITC-2019 \and Integer Linear Programming \and University Timetabling \and Local Search}
% \PACS{PACS code1 \and PACS code2 \and more}
% \subclass{MSC code1 \and MSC code2 \and more}
\end{abstract}

\section{Preprocessing}

Our approach utilizes three pre-processing methods to improve performance of the overall method.
\begin{itemize}
\item Find independent sub-problems of classes
\item Remove invalid possible time-room pairs from the domain of class, due to unavailable rooms.
\item Merge students with the same course enrollemnt.
\item Compute possible room conflict beforehand by clustering class with conflicting time-room pairs.
\end{itemize}

While parcing the XML file, we create for each class a set of possible domains ($D$). This set is a pair of time-room avaible for this class. Only valid time-room pairs are genareted, \emgh{i.e.} the room unavailablity constraints are dealt here. The process can, in the best scenarius reduce in 10\% the size of the domain of a class.
This type of process could be replicated to "same" type constraints (unit propoagation style). However, the overhand genereted by the process does able to improve the overall performance.

\section{Integer Linear Programming}

We propose an Integer Linear Programming solve the ITC-2019 problem. An in-depth description of the models is presented below.

The schedule of a class (day and time slots) and its assigment to a room is represented by the variable  $x$, where  $x_c,d$ equals 1 if and only if a class $c$  is assigned the pair $d \in D$. This variable reduces the complexity of the problem for Cplex solver.

The encoding can be genareted while parcing the input file.
\section{Local Search}
\label{intro}
Your text comes here. Separate text sections with
\section{Section title}
\label{sec:1}
Text with citations% \cite{Lemos} and \cite{RefJ}.
\subsection{Subsection title}
\label{sec:2}
as required. Don't forget to give each section
and subsection a unique label (see Sect.~\ref{sec:1}).
\paragraph{Paragraph headings} Use paragraph headings as needed.
\begin{equation}
a^2+b^2=c^2
\end{equation}

% For one-column wide figures use
\begin{figure}
% Use the relevant command to insert your figure file.
% For example, with the graphicx package use
  \includegraphics{example.eps}
% figure caption is below the figure
\caption{Please write your figure caption here}
\label{fig:1}       % Give a unique label
\end{figure}
%
% For two-column wide figures use
\begin{figure*}
% Use the relevant command to insert your figure file.
% For example, with the graphicx package use
  \includegraphics[width=0.75\textwidth]{example.eps}
% figure caption is below the figure
\caption{Please write your figure caption here}
\label{fig:2}       % Give a unique label
\end{figure*}
%
% For tables use
\begin{table}
% table caption is above the table
\caption{Please write your table caption here}
\label{tab:1}       % Give a unique label
% For LaTeX tables use
\begin{tabular}{lll}
\hline\noalign{\smallskip}
first & second & third  \\
\noalign{\smallskip}\hline\noalign{\smallskip}
number & number & number \\
number & number & number \\
\noalign{\smallskip}\hline
\end{tabular}
\end{table}


%\begin{acknowledgements}
%If you'd like to thank anyone, place your comments here
%and remove the percent signs.
%\end{acknowledgements}


% Authors must disclose all relationships or interests that 
% could have direct or potential influence or impart bias on 
% the work: 
%
\section*{Conflict of interest}
%
The authors declare that they have no conflict of interest.




\newacronym{kul}{KUL}{Katholieke Universiteit Leuven}
\newacronym{asp}{ASP}{Answer Set Programming}
\newacronym{mpp}{MPP}{Minimal Perturbation Problem}
\newacronym{ilp}{ILP}{Integer Linear Programming}
\newacronym{ist}{IST}{Instituto Superior T\'ecnico}
\newacronym{cp}{CP}{Constraint Programming}
\newacronym{itc}{ITC}{International Timetabling Competition}



\bibliographystyle{spbasic} \bibliography{cite

\end{document}
% end of file template.tex

